\documentclass[aspectratio=169]{beamer}
% setup for the handout
\usepackage[subpreambles=true]{standalone}
\usepackage{import}
% language and font set up
\usepackage[utf8]{inputenc}
\usepackage{amsmath}
\usepackage{amsfonts}
\usepackage{mathrsfs}
\usepackage{tikzsymbols}
\usepackage{textcomp}
% additonal packages
\usepackage{graphicx}
\usepackage{color}
\usepackage{hyperref}
% listings
\usepackage{listings}
\definecolor{mygreen}{rgb}{0,0.6,0}
\definecolor{mygray}{rgb}{0.5,0.5,0.5}
\definecolor{mymauve}{rgb}{0.58,0,0.82}

\lstset{
  backgroundcolor=\color{white},   % choose the background color; you must add \usepackage{color} or \usepackage{xcolor}; should come as last argument
  basicstyle=\footnotesize,        % the size of the fonts that are used for the code
  breakatwhitespace=false,         % sets if automatic breaks should only happen at whitespace
  breaklines=true,                 % sets automatic line breaking
  captionpos=b,                    % sets the caption-position to bottom
  commentstyle=\color{mygreen},    % comment style
  deletekeywords={...},            % if you want to delete keywords from the given language
escapeinside={\%*}{*)},          % if you want to add LaTeX within your code
extendedchars=true,              % lets you use non-ASCII characters; for 8-bits encodings only, does not work with UTF-8
firstnumber=1,                   % start line enumeration with line 1000
frame=single,                    % adds a frame around the code
keepspaces=true,                 % keeps spaces in text, useful for keeping indentation of code (possibly needs columns=flexible)
keywordstyle=\color{blue},       % keyword style
language=TeX,                    % the language of the code
morekeywords={*,...},            % if you want to add more keywords to the set
numbers=none,                    % where to put the line-numbers; possible values are (none, left, right)
numbersep=5pt,                   % how far the line-numbers are from the code
numberstyle=\tiny\color{mygray}, % the style that is used for the line-numbers
rulecolor=\color{black},         % if not set, the frame-color may be changed on line-breaks within not-black text (e.g. comments (green here))
showspaces=false,                % show spaces everywhere adding particular underscores; it overrides 'showstringspaces'
showstringspaces=false,          % underline spaces within strings only
showtabs=false,                  % show tabs within strings adding particular underscores
stepnumber=2,                    % the step between two line-numbers. If it's 1, each line will be numbered
stringstyle=\color{mymauve},     % string literal style
tabsize=2,                       % sets default tabsize to 2 spaces
title= title                     % show the filename of files included with \lstinputlisting; also try caption instead of title
}

% inserting picuters
\newenvironment{MyPicture}[4]{
  \begin{figure}[h!]
    \centering
    \includegraphics[#4]{img/#1}
    \caption{#2}
    \label{fig:#3}
  }
{\end{figure}}
% design
\usetheme{Berlin}
% \usecolortheme{whale}
% informations
\title{How to Vim up your \LaTeX}
\author{Lukas Wais 11816105 SKZ: 033 521}
\institute{Wissenschaftliches Schreiben und Layouten anhand von \LaTeX2}
\date{\today}
\begin{document}
% title frame
\frame{\titlepage}
% table of content
\begin{frame}[allowframebreaks]{Table of Contents}
  \tableofcontents
\end{frame}
% introduction
\section{Introduction}
\subsection{What is Vim?}
\begin{frame}{This is Vim}
  \begin{MyPicture}{vim}{Vim}{Vim}{width=0.4\textwidth}
  \end{MyPicture}
\end{frame}
\begin{frame}{Vim is hard}
  \begin{MyPicture}{meme}{}{meme}{width=0.4\textwidth}
  \end{MyPicture}
\end{frame}
\begin{frame}{This is Vim}
  \begin{itemize}
    \item Vim $=$ Vi improved, it is an improvment over the old editor Vi
    \item It does have three modes \textbf{normal, visual and insert}
    \item It does have a really hard learning curve, but it is worth it!
    \item Configure it with the vimrc file
  \end{itemize}
\end{frame}
\subsection{What do we need?}
\begin{frame}{Requirements}
  Please keep in mind: those are only my recommendations. There are various other possibilities.
  \begin{itemize}
    \item An Unix like operating system. For running Vim natively.
    \item vim-latex, the \LaTeX~plugin.
    \item A PDF reader which supports PDF-TeX Sync. You can use Skim for example.
    \item gVim
  \end{itemize}
\end{frame}
% Working with Vim
\section{Working with Vim}
\subsection{Compiling with Vim}
\begin{frame}[fragile]{Compile}
  The fastest way to this is to use one of many vim-latex shortcuts
  \begin{itemize}
    \item $\backslash$ll $\ldots$ two lower case L
    \item You can also set compilation rules. Either with menu TeX-Suit or with the command g:Tex\_CompileRule\_$<$format$>$
    \item Take a look here:\\ \url{http://vim-latex.sourceforge.net/documentation/latex-suite/compiler-rules.html}
  \end{itemize}
\end{frame}
\subsection{Navigating Code}
\begin{frame}{Use the Power of Vim to navigate through your code}
  \begin{itemize}
    \item Just use the normal Vim commands to move around.
    \item h $\ldots$ left
    \item j $\ldots$ down
    \item k $\ldots$ up
    \item l $\ldots$ right
  \end{itemize}
  \vspace{0.5cm}
  Those letters may seem arbitary, but just take a look at your keyboard \Winkey
\end{frame}
\begin{frame}{Jump Points}
  vim-latex uses jump points. Everytime you generate something with vim-latex it creates those points. It may be disturbing at the beginning, since you are getting used to it, it becomes really handy
  \begin{itemize}
    \item You can jump there in normal mode or even the insert mode.
    \item The shortcut is \textbf{CTRL + j}
  \end{itemize}
\end{frame}

\begin{frame}[fragile]{Jump Points}
  \begin{lstlisting}[caption = Jump Points]
  \begin{figure}[<+htpb+>]
    \centering
    \includegraphics{<+file+>}
    \caption{<+caption text+>}
    \label{fig:<+label+>}
  \end{figure}<++>
  \end{lstlisting}
\end{frame}

\subsection{Helpful Shortcuts}
\begin{frame}{A List of handy Shortcuts}
  Just type them in, they even work directly in the insert mode.
  \vspace{0.2cm}
  \begin{itemize}
    \item SSE for section
    \item SSS for subsection
    \item more section mappings can be found here: \\ \url{http://vim-latex.sourceforge.net/documentation/latex-suite/section-mappings.html}
    \item EFI for images
    \item All mappings can be found here: \\ \url{http://vim-latex.sourceforge.net/documentation/latex-suite.html}
  \end{itemize}
\end{frame}
% Setup Guide
\section{Setup Guide}
\subsection{Setup}
\begin{frame}{Recommended Setup}
  \begin{itemize}
    \item gVim
    \item Vundle
    \item Skim
    \item vim-latex
  \end{itemize}
\end{frame}
\subsection{Installation}
\begin{frame}{Installation Steps}
  \begin{enumerate}
    \item Install gVim
    \item Setup Vundle
    \item Edit your .vimrc file
    \item Install vim-latex
    \item Start working
  \end{enumerate}
\end{frame}
\begin{frame}{$\sim$/.vimrc}
  \begin{MyPicture}{vimrc}{vimrc}{vimrc}{width=0.5\textwidth}
  \end{MyPicture}
\end{frame}
\subsection{Guide}
\begin{frame}{Download the Complete Guide}
  \begin{itemize}
    \item You can find a complete installation guide in the paper on Github.
      \item \url{https://github.com/x21L/vim-up-your-latex}
  \end{itemize}
\end{frame}
% Vim vs. TeXstudio
\section{Vim vs. TeXstudio}
\subsection{Pros}
\begin{frame}{Vim over TeXstudio}
  \begin{itemize}
    \item Really fast
    \item Great UI
    \item Extremly customizable
    \item Nearlly unlimited possibilities
    \item Versioning vimrc files with git $\rightarrow$ same config on different systems
  \end{itemize}
\end{frame}
\subsection{Cons}
\begin{frame}{TeXstudio over Vim}
  \begin{itemize}
    \item Feature rich
    \item Easy to use
    \item Download and go
    \item Runs easily on every major system
  \end{itemize}
\end{frame}
% Outlook
\section{Outlook}
\subsection{Alternatives to vim-latex}
\begin{frame}{}
  \begin{itemize}
    \item vimtex
    \item Just Vim with latexmk
    \item latex-box
    \item Take a look at \url{https://vimawesome.com/?q=latex}
  \end{itemize}
\end{frame}
\subsection{Download it from Github}
\begin{frame}{}
  \begin{itemize}
    \item \url{https://github.com/x21L/vim-up-your-latex}
    \item You can also find an A4 version of the slides there.
    \item[]
    \item You can exit and save with \textbf{:wq}
  \end{itemize}
\end{frame}
\subsection{Live Demo}
\begin{frame}{}
  \centering{\huge Demo Time}
\end{frame}
\begin{frame}{}
  \begin{MyPicture}{questions}{Questions}{Questions}{width=0.3\textwidth}
  \end{MyPicture}
\end{frame}
\end{document}
